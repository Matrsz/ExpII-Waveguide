\documentclass[11pt,a4paper]{article}
\usepackage[utf8]{inputenc}
\usepackage[spanish]{babel}
\usepackage{amsmath}
\usepackage{amsfonts}
\usepackage{amssymb}
\usepackage{graphicx}
\usepackage[left=1.2cm,right=1.2cm,top=2cm,bottom=2cm]{geometry}
\date{\small{\today}}
\usepackage{fancyhdr}
\usepackage{afterpage}
\usepackage{titlesec}
\usepackage{float}
\usepackage{gensymb}
\usepackage{xfrac}
\usepackage{tabularx}
\usepackage{multicol}
\usepackage[font=small]{caption}
\usepackage{scrextend}
\usepackage[toc,page]{appendix}
\usepackage{tikz}
\usepackage{tkz-euclide}
\usepackage[siunitx,americanvoltages,fulldiodes]{circuitikz}
\usepackage{stackengine}
\usepackage{mathtools}
\DeclarePairedDelimiter\abs{\lvert}{\rvert}%
\DeclarePairedDelimiter\norm{\lVert}{\rVert}%

\renewcommand\appendixpagename{Apéndices}
\renewcommand\appendixname{Apéndice}

\DeclareMathOperator{\arctantwo}{arctan2}

\titleformat{\section}{\Large\bfseries}{}{0em}{}[]
\titleformat{\subsection}{\large\bfseries}{}{0em}{}[]
\titleformat{\subsubsection}{\bfseries}{}{0em}{}[]
\titleformat{\chapter}{\large\bfseries}{}{0em}{}[]


\setlength\parindent{0pt}


\begin{document}
\title{Adaptación de Impedancia en Guía de Onda Rectangular}
	\LARGE{\textsc{Laboratorio II}}\\
	\Large{Adaptación de Impedancia en Guía de Onda Rectangular}\\
\begin{large}
\small\textsc{Bercic, Jerónimo}\\
\small\textsc{Roqueta, Matías Daniel}\\
\small{Instituto Balseiro, Centro Atómico Bariloche, Comisión Nacional de Energía Atómica}\\
\end{large}
\setcounter{page}{1}

\lhead{Laboratorio II}%Materia
\rhead{Adaptación de Impedancia en Guía de Onda Rectangular}%Título 
\chead{}

\lfoot{J. Bercic, M. Roqueta}
\cfoot{Instituto Balseiro} 
\rfoot{\thepage} 
\renewcommand{\headrulewidth}{0.4pt} 
\renewcommand{\footrulewidth}{0.4pt} 
\pagestyle{fancy}

\hrule
\normalsize
\section{Resumen}

\begin{multicols}{2}
\section{Introducción}
Una guía de ondas electromagnéticas es un dispositivo que se utiliza para restringir la energía de una onda, y que ésta sea propagada en una dirección deseada. De esta manera se puede aprovechar dicha energía de manera más eficiente. Su utilidad es evidente, por ejemplo, en la fabricación de antenas, donde se busca que la propagación sea en una dirección específica y que las pérdidas de información sean mínimas. En la Figura 1 se muestra una guía de ondas electromagnéticas rectangular.
%\begin{center}
%    \includegraphics[scale=0.4]{Images/Imagen1.jpg} \\
%    \textbf{Figura 1}. Guía de ondas electromagnéticas rectangular.
%\end{center}
Dentro del contexto de éste tipo de guía de ondas, se encuentran los que se conocen como modos TE$_{mn}$ (Transverse Electric Field) y TM$_{mn}$ (Transverse Magnetic Field), que se refieren a si la onda transversal a la dirección de propagación es la eléctrica o la magnética, respectivamente, $m$ es el número de medias-ondas a lo largo de la guía, y $n$ es el número de medias-ondas a lo alto de la guía. En la Figura 2 se presenta un esquema de una guía de ondas cuyo modo es el TE$_{10}$.
%\begin{center}
%    \includegraphics[scale=0.15]{Images/TE10.jpg} \\
%    \textbf{Figura 2}. Guía de ondas electromagnéticas con modo TE$_{10}$.
%\end{center}
El modo de propagación dependerá tanto de las dimensiones de la guía, así como de la frecuencia de la onda propagada. \\ \\
Para poder caracterizar la guía de ondas, se puede pensar en ella como una línea de transmisión, o sea, una cascada de cuadripolos (Figura 3) distribuidos a lo largo de la línea.
\begin{center}
\begin{tikzpicture}[scale=0.75]
\draw (1,3) to [L, l_=$L\:dz$] (-2,3);
\draw (1,3) to [C, l_=$C$] (1,1);
\draw[ultra thin, fill=pink,fill opacity=0.15] (-3,0.25) rectangle (2,3.8);
\draw[->] (-2,1) -- (3,1);
\draw[->] (-2,3) -- (3,3);
\draw (-4,1) -- (-2,1);
\draw (-4,3) -- (-2,3);
\draw (-2,3) -- (-2,1);
\node at (-7,1) { };
\draw[->] (-5.5,3) -- node[above] {$i(z,t)$} (-4.2,3);
\draw[->] (3.2,3) -- node[above] {$i(z+dz,t)$} (4.4,3);
\draw[->] (-4.2, 1.2) -- node[left] {$v(z,t)$} (-4.2,2.7);
\draw[->] (3.2, 1.2) -- node[right] {$v(z+dz,t)$} (3.2,2.7);
\end{tikzpicture}
\textbf{Figura 3}: Esquema del modelo de los cuadripolos a disponer en cascada.
\end{center}
Al igual que una línea de transmisión, las guías de ondas cuentan con una impedancia característica $Z_0$, la cual se puede calcular sabiendo la frecuencia $f$ a la que se propaga la onda, de la siguiente forma
\begin{equation}
    Z_0=\frac{Z_\text{vacío}}{\lambda_\text{vacío}}=Z_\text{vacío}\frac{f}{c}
\end{equation}
Donde $Z_\text{vacío}$ es la impedancia característica del vacío, igual a $376.6$ $\Omega$, $\lambda_\text{vacío}$ es la longitud de la onda propagada si ésta se propagase en el vacío, y $c$ es la velocidad de la luz.\\ \\
Sabiendo la impedancia característica de la guía de ondas, se puede adaptar a la impedancia del plano de carga $Z_\text{L}$, y así poder aprovechar de manera más eficiente la energía transmitida. Éste será el caso si $Z_0$ es igual a $Z_\text{L}$. Para saber si la guía está adaptada se utiliza el \textbf{coeficiente de reflexión} $\Gamma$,
\begin{equation}
    \Gamma = \frac{Z_\text{L}-Z_0}{Z_\text{L}+Z_0}
\end{equation}
el cual vale $\Gamma = -1$ para un corto circuito, $\Gamma = 1$ para un circuito abierto, y $\Gamma = 0$ para la guía adaptada. \\ \\ 
Otra forma de verificar si la guía está adaptada, es analizando la onda estacionaria que se forma debido a la reflexión de la onda en el plano de carga (Figura 4).
%\begin{center}
%    \includegraphics[scale=0.28]{Images/Para el informe.jpg} \\
%    \textbf{Figura 4}: Onda estacionara formada en la guía debido a la reflexión en el plano de carga. 
%\end{center}
Para ello se define el ROE (razón de onda estacionaria) como
\begin{equation}
    \text{ROE} = \frac{1+\abs{\Gamma}}{1-\abs{\Gamma}}
\end{equation}
Este número también se puede calcular haciendo el cociente entre las amplitudes máxima y mínima de la onda estacionaria, y tiene un rango de 
$$
1<\text{ROE} < \infty
$$
Cuanto se quiere adaptar la guía, el ROE debe tender a 1. \\ \\
En este laboratorio, se busca medir las propiedades de la onda estacionaria que se forma bajo distintas condiciones del plano de carga. Además, se busca adaptar la impedancia para ciertas condiciones. 
\section{Método Experimental}

Primero se determina la frecuencia de emisión del diodo emisor $f$, se construye una línea de transmisión con guía de onda rectangular conectando en cascada los siguientes elementos, ilustrada en la figura \ref{fig:arr1}
\begin{multicols}{2}
    \begin{labeling}{III} 
        \item [I] Emisor Diodo Gunn
        \item [II] Aislador
        \item [III] Atenuador 3 dB
        \item [IV] Cavidad Resonante
        \item [V] Antena Detectora
        \item [VI] Cortocircuito Móvil
    \end{labeling}        
\end{multicols}
\begin{figure}[H]
    \centering
    \includegraphics[width=\linewidth]{Images/arreglo1.pdf}
    \caption{Línea de transmisión para medir frecuencia de emisión. La cavidad resonante actúa como filtro, absorbiendo su frecuencia de resonancia y dejando pasar otras frecuencias.}
    \label{fig:arr1}
\end{figure}
Se ajusta el cortocircuito móvil hasta detectar un máximo de onda estacionaria con la antena detectora, y se procede a ajustar la frecuencia de resonancia de la cavidad hasta que esta coincida con la frecuencia de emisión del diodo Gunn.\\

Conociendo la frecuencia de operación del emisor, se procede a caracterizar la longitud de onda dentro de la guía $\lambda_g$ y su impedancia intrínseca $Z_0$. 
\begin{figure}[H]
    \centering
    \includegraphics[width=\linewidth]{Images/arreglo2.pdf}
    \caption{Linea de transmisión para medir longitud de onda. La terminación en cortocircuito provoca una onda estacionaria en la guía, la onda estacionaria se mide con una antena detectora en una línea ranurada.}
    \label{fig:arr2}
\end{figure}

Una vez conocidas $\lambda_g$ y $Z_0$ se pueden adaptar impedancias. Se costruye la línea de transmisión correspondiente a la figura \ref{fig:arr4}.

\begin{figure}[H]
    \centering
    \includegraphics[width=\linewidth]{Images/arreglo4.pdf}
    \caption{Sistema de adaptación de impedancias para una impedancia de carga $Z_L$, un tornillo capacitivo actúa como stub en paralelo.}
    \label{fig:arr4}
\end{figure}

Se sigue el siguiente procedimiento para la adapatación de impedancia

\begin{enumerate}
    \item Se mide la onda estacionaria con $Z_L$ en cortocircuito, referenciando el plano de carga a un mínimo de onda estacionaria.
    \item Se cambia $Z_L$ por una impedancia arbitraria y se mide la onda estacionaria. Se calculan la ROE y el desplazamiento del plano de carga.
    \item Usando la ROE y el desplazamiento del plano de carga para calcular la posición del adaptado $\ell$ y su admitancia $y_a$. Estos cálculos se hacen con el diagrama de Smith y numéricamente con un script Python, validando el resultado.
    \item Se ajusta la posición y admitancia del adaptador, se mide la onda estacionaria validando la disminución de la ROE ante impedancia adaptada.
\end{enumerate}

Este método de adaptación de impedancia se usa para adaptar un sistema de comunicaciones transmisor-receptor, representado en la figura \ref{fig:sistema}

\begin{figure}[H]
    \centering
    \includegraphics[width=\linewidth]{Images/sistema.pdf}
    \caption{Sistema transmisor-receptor.
    \texttt{Tx:} línea de la figura \ref{fig:arr4} con antena de bocina en $Z_L$. 
    \texttt{Rx:} detector de señal adaptado con una antena de bocina.}
    \label{fig:sistema}
\end{figure}

El sistema se adapta con el siguiente procedimiento
\begin{enumerate}
    \item Se retira el módulo Rx y se mide la onda estacionaria, observando la ROE y validando que la antena de bocina adapta la impedancia de la guía a la impedancia del espacio libre.
    \item Se ubica el módulo Rx a 10 cm del módulo Tx. Se mide la ROE, observando el incremento de la misma respecto a la medición anterior.
    \item Se realiza el procesdimiento de adaptación de impedancia en presencia del módulo Rx.
\end{enumerate}

Se registra la intensidad de señal medida por Rx antes y después de adaptar impedancias, verificando la pérdida en potencia transmitida por impedancia desadaptada.

\section{Resultados}


\section{Discusión}

\section{Conclusiones}

\bibliography{LockIn}
\bibliographystyle{unsrt}

\end{multicols}
\newpage
\begin{appendices}
\vspace{-1em}
\hrule
\vspace{1em}
\normalsize
\section{Apéndice 1 -}
\end{appendices}

\begin{multicols}{2}

\end{multicols}

\end{document}